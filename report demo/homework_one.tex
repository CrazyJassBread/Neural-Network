% !TEX program = xelatex
% -------------------------------------------------------
\documentclass[nols, a4paper]{tufte-handout}

% --- 基础设置与宏包 ---
\usepackage{graphicx}
\usepackage{booktabs}
\usepackage{amsmath}
\usepackage{lipsum}
\usepackage{hyperref}
\hypersetup{colorlinks, allcolors=blue}

% 设置全文行距为 1.5 倍 ---
\usepackage{setspace}

% 自定义标题样式并强制编号 ---
\usepackage{titlesec}

% 设置编号深度
\setcounter{secnumdepth}{3} 

% 格式化一级标题 ,在标题后加横线
\titleformat{\section}[display]
  {\normalfont\Large\bfseries}
  {\thesection}
  {1em}
  {}
  [\vspace{1ex}\hrule]

% 格式化二级标题
\titleformat{\subsection}
  {\normalfont\large\bfseries}
  {\thesubsection} % <-- 加入编号
  {1em}
  {}

% 格式化三级标题 (paragraph)
\renewcommand{\theparagraph}{\thesubsection.\arabic{paragraph}}
\titleformat{\paragraph}
  {\normalfont\normalsize\bfseries}
  {\theparagraph} % <-- 加入编号
  {1em}
  {}

% 字体设置 
\usepackage{xeCJK}
\setmainfont{Times New Roman}
\setsansfont{Arial} 
\setmonofont{Courier New}
\setCJKmainfont{STFangsong}
\setCJKsansfont{STHeiti}
\setCJKmonofont{STFangsong}

% 解决图表标题的中文问题
\usepackage{caption}
% 全局设置:表格标题在上方,图片标题在下方。
\captionsetup[table]{position=top}
\captionsetup[figure]{position=bottom}
\captionsetup{font=small, labelfont=bf}


% ----------------------------------------------------------------------
% 页面布局调整 (核心部分)
% ----------------------------------------------------------------------
\usepackage{geometry}
\geometry{
  a4paper,
  left=20mm,
  top=35mm, 
  bottom=25mm,
  textwidth=113mm,
  marginparwidth=51mm,
  marginparsep=5mm
}

% --- 文档信息 ---
\title{Neuron modeling discussion}
\author{独宇涵}
\date{\today}

\begin{document}

% 应用 1.5 倍行距
\onehalfspacing 

\maketitle

\tableofcontents

% --- 优化 2: 在目录后添加分界线 ---
\bigskip
\noindent\rule{\linewidth}{0.4pt}
\bigskip

\section{模板使用指南}

\subsection{基本文本与边注} 
这是正文部分的第一段。`tufte-latex` 宏包提供了一个非常优雅的笔记记录方式。正文在页面的左侧,而补充说明、引申思考、或者相关的引用可以被放置在右侧的边注区域。这种方式可以让主线内容保持清晰连贯,同时又不丢失重要的辅助信息。\sidenote{这是一个边注(sidenote)的示例。它会自动与正文中标记的位置在垂直方向上对齐。} 正如您所见,这个边注就出现在了这段话的旁边。

我们可以继续书写正文内容。\lipsum[1][1-4] 这里的拉丁文仅用于填充版面,以展示长段落的排版效果。\sidenote{这里是第二个边注,用来解释 `lipsum` 的作用。`lipsum` 宏包可以方便地生成无意义的占位文本。} 所有的字体都已经按照您的要求进行了设置:中文是仿宋,而 English text in the paragraph uses Times New Roman.

\subsection{图表示例} 

图表是笔记中不可或缺的元素。`tufte-latex` 提供了特殊的浮动体环境,可以将图表也放置在边注区域。

\paragraph{边注图片}
下面是一个将图片放置在右侧边注区的例子。当图片或表格较小,适合作为补充说明时,这种方式非常有用。它不会打断正文的阅读流。我们在这段话的末尾引用这张图,见图 \ref{fig:margin-fig}。

\begin{marginfigure}
  \includegraphics[width=\linewidth]{example-image-a} 
  % 图片的标题在下方,是正确的
  \caption{这是一个放置在边注区的图片。它的宽度被自动设置为边注栏的宽度。}
  \label{fig:margin-fig}
\end{marginfigure}

正文可以继续。即使插入了边注图片,正文的排版也完全不受影响,这正是其设计的精妙之处。\lipsum[2][1-3]

\paragraph{通栏图片} 
如果图片非常重要,或者尺寸较大,不适合放在边注区,我们可以使用通栏(full width)的图片。这需要使用带星号的 `figure*` 环境。

\begin{figure*}
  \includegraphics[width=\linewidth]{example-image-b}
  % 图片的标题在下方,是正确的
  \caption{这是一个通栏图片,它会横跨正文和边注两个区域。适合展示重要或宽幅的图像。}
  \label{fig:full-fig}
\end{figure*}


\section{表格示例}

与图片类似,表格也可以被放置在边注区或者通栏显示。

\subsection{边注表格}

小型的总结性表格非常适合放在边注区,如下表示例(表 \ref{tab:margin-tab})。
\begin{margintable}
  % --- 核心修改:将标题和标签移到表格数据的上方 ---
  \caption{一个简单的边注表格。}
  \label{tab:margin-tab}
  \centering
  \begin{tabular}{l c}
    \toprule
    项目 & 数值 \\
    \midrule
    参数 A & 10.5 \\
    参数 B & 22.1 \\
    参数 C & 7.8  \\
    \bottomrule
  \end{tabular}
\end{margintable}
我们可以在正文中继续讨论表格中的数据,而读者可以很方便地同时看到数据和分析。例如,参数 A 和参数 C 的和远小于参数 B。

\subsection{正文宽度表格}

对于需要在主文本区域内显示,并且宽度与正文一致的表格,可以使用标准的 `table` 环境,并结合 `tabular*` 来精确控制宽度。如表 \ref{tab:text-width-tab} 所示。

\begin{table}[h!]
    % 这个表格的标题位置本来就是正确的
    \caption{一个与正文等宽的三线表。}
    \label{tab:text-width-tab}
    \begin{tabular*}{\textwidth}{@{\extracolsep{\fill}} l c c r}
        \toprule
        产品类别 & 季度一销量 & 季度二销量 & 增长率 (\%) \\
        \midrule
        电子产品 & 1,200 & 1,500 & 25.0 \\
        家居用品 & 850 & 920 & 8.2 \\
        图书音像 & 2,100 & 1,950 & -7.1 \\
        \bottomrule
    \end{tabular*}
\end{table}

\lipsum[3] % 再来一段占位文字作为结尾。

\end{document}